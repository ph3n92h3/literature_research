\usepackage{makeidx}
\makeindex

\usepackage{latexcolors}

% 背景图片
\usepackage{tikz}
\pgfdeclareimage[height=\paperheight,width=\paperwidth]{myimage}{figures/Kohrvirab.jpg}
\usebackgroundtemplate{\tikz\node[opacity=0.1] {\pgfuseimage{myimage}};}

\usepackage{multicol}   %目录分栏

\usepackage[space, hyperref, UTF8]{ctex}	%中文支持

% 数理
\usepackage{amsmath,amssymb,amsthm}
\usepackage{esint}		% \oiint
\usepackage{amsfonts}
\usepackage{latexsym}
\usepackage{extarrows}  	%长等号上写字\xlongequal{文字}
\usepackage{physics}
\usepackage{upgreek}    	%直立的希腊字母\upalpha
\usepackage{bm}     		%加粗\bm{}
\usepackage{slashed}    	%Dirac slash
\usefonttheme{professionalfonts}    % 数学字体

%插入 Mathematica 代码请先在当前目录下放入 mmacells.sty
%\usepackage{mmacells}
%\usepackage[framed,numbered,autolinebreaks,useliterate]{mcode}  %MATLAB 官方宏包
% \usepackage{listings}		% 插入代码:已过时

\usepackage{graphicx}   %插入图片的宏包
\usepackage{float}      %设置图片浮动位置的宏包
\usepackage{subfigure}  %插入多图时用子图显示的宏包

%\usepackage[xllnames]{xcolor}

%\usepackage[hidelinks]{hyperref}

%插入 emoji 表情:使用 \LuaLaTeX,\Memoji{🎄}
%\usepackage{metalogo}
%\usepackage{fontspec}
%\setmainfont{Comic Sans MS}
%\newfontfamily\emojifont{Segoe UI Emoji}[Renderer=Harfbuzz]
%\newcommand{\emoji}[1]{\emojifont #1}
%\newcommand{\Memoji}[1]{\ifmmode \text{\emoji{#1}} \else \emoji{#1}\fi}

\usetheme{Goettingen}
\usecolortheme{crane}

\title[文献调研]{2022文献调研}
\subtitle{光电器件中的波长选择}
\author{冯哲}
\date{\today}
\institute{河海大学 理学院}
%\titlegraphic{\includegraphics[width=0.17\textwidth]{ias.pdf}}