\begin{frame}[c]
    \frametitle{理想的波长选择仪器的性质}
    \begin{columns}
        \begin{column}{.5\textwidth}
            \begin{itemize}
                \item 最小的可调时间
                \item 最小的带外传输
                \item 最小物理厚度
                \item 低功耗
                \item 对光的极化不敏感
                \item 可选择带通
                \item 完美的调制传递函数(MTF)
            \end{itemize}
        \end{column}
        \begin{column}{.5\textwidth}
            \begin{itemize}
                \item 对环境不敏感(如温度、湿度)
                \item 对入射光的入射角度不敏感(宽视场)
                \item 宽的光谱范围
                \item 透射峰高
                \item 大光圈
                \item 连续的通带
                \item 对波长的随机访问
            \end{itemize}
        \end{column}
    \end{columns}
\end{frame}

\begin{frame}[c]
    \frametitle{Application of wavelength selection}
    \begin{itemize}
        \item 光电器件
        \item (化学/生物)成分分析:\begin{itemize}
                  \item 制药工业、化学工业过程监控
                  \item 农业、生物学
                  \item 医疗诊断和保健、法医
                  \item 大气环境
              \end{itemize}
        \item 地球遥感
        \item 可见光通信
        \item 色彩视觉、艺术修复、考古
    \end{itemize}
\end{frame}

\begin{frame}[c]
    \frametitle{Acknowledgement}
    \Large{\begin{center}
            衷心感谢各位老师指正和支持!\\
            ~\\
            感谢团队和合作者!\\
            ~\\
            感谢河海大学理学院!\\
        \end{center}}
\end{frame}

\printindex
\end{document}